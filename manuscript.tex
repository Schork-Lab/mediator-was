\documentclass[11pt]{article}
\usepackage{amsmath}
\usepackage{graphicx}
\usepackage{longtable}
\usepackage{float}
\usepackage{fontspec}
\usepackage{xunicode}
\usepackage[backend=biber,date=year,doi=false,isbn=false,style=nature,url=false]{biblatex}
\usepackage[xetex]{hyperref}
\usepackage{rotating}
\usepackage{parskip}
\usepackage[letterpaper]{geometry}

\author{Kunal Bhutani, Abhishek Sarkar, Nicholas J. Schork, Manolis Kellis}
\date{}
\title{Modeling prediction error improves power of transcriptome-wide association studies}

\addbibresource{manuscript.bib}

\begin{document}

\maketitle
Thousands of loci associated with hundreds of complex diseases have been
reported in the NHGRI catalog of genome-wide association studies
\cite{10.1093/nar/gkt1229} (GWASs). However, most genome-wide significant loci
are devoid of protein-coding alterations \cite{10.1073/pnas.0903103106} and
likely instead affect transcriptional regulation. Several studies have directly
investigated the role of transcriptional regulation on complex diseases by
jointly considering genotypes, expression, and phenotypes \cite{xxx}. However,
such studies require all data to be measured in all samples, which is still
prohibitive at the scale of GWAS.

Recent large-scale efforts have produced reference data on molecular phenotypes
relevant to transcriptional regulation. The Gene-Tissue Expression Project
(GTEx) has generated XXX reference transcriptomes
\cite{10.1126/science.1262110}. Several methods have been proposed to use these
reference data to impute gene expression into GWAS cohorts where expression is
not measured \cite{10.1038/ng.3367,10.1101/024083}. From imputed expression, we
can conduct transcriptome-wide association studies (TWASs) to directly find
genes whose expression is associated with phenotype.

Here, we develop a method to explicitly model error in imputed expression and
propagate this error through TWAS. We show through simulation and application
to real data that our method improves power to detect genes associated with
phenotype.

\section{Results}

\subsection{Simulation study}

\subsection{Application to WTCCC diseases}

\section{Discussion}

\section{Methods}

\subsection{Additive measurement error model}

We assume a continuous phenotype \(y_i\) on n individuals and regress phenotype
on predicted expression \(w_i\) for each gene. For ease of exposition, we first
describe a model with no additional covariates. The results readily apply to
generalized linear models, so we use logistic regression for binary phenotypes.

\begin{align*}
  y_i &= \beta_0 + \beta_1 w_i + \epsilon_i\\
  \epsilon_i &= \text{error in equation}\\
  y_i &= \text{phenotype of individual $i$}\\
  w_i &= \text{predicted expression of individual $i$}\\
\end{align*}

We assume additive measurement error on the predicted expression value:

\begin{align*}
  w_i &= x_i + u_i\\
  x_i &= \text{true expression (latent) of individual $i$}\\
  u_i &= \text{error in predicted expression of individual $i$}\\
\end{align*}

We assume measurement errors have mean 0 and finite variance:

\begin{align*}
  E[u_i] &= 0\\
  V[u_i] &= \sigma_{ui}^2\\
\end{align*}

Under these assumptions, the naive estimators \(\hat\beta_0\) and \(\hat\beta_1\)
are biased because the error in equation becomes correlated with the predicted
expression:

\begin{equation}
  y_i = \beta_0 + \beta_1 x_i + (\epsilon_i - \beta_1 u_i)
\end{equation}

Specifically, \(\hat\beta_1\) is biased towards zero by a multiplicative factor,
which leads to loss of power to detect significant effects on phenotype:

\begin{align*}
  \hat\beta_1 &= \frac{\sigma_x^2}{\sigma_x^2 + \sigma_u^2} \beta_1 = \kappa \beta_1\\
  \sigma_x^2 &= \text{variance of true expression}\\
  \sigma_u^2 &= \text{variance of errors in predicted expression}\\
\end{align*}

Given \(\hat\sigma_{ui}^2\), we can correct the naive estimates using regression
calibration. We regress \(y_i\) on \(\hat x_i = \bar w - \hat\kappa (w_i - \bar
w)\), yielding estimates \(\hat\beta^*\). We describe how to estimate
\(\hat\sigma_{ui}^2\) below.

To estimate an association \(p\)-value, we perform a Wald test:

\begin{align*}
  \theta &= \left(\frac{\hat\beta_1^*}{SE(\hat\beta_1^*)}\right)^2 \sim \chi^2(\theta; 1)\\
  H_0&: \theta = 0\\
  H_1&: \theta > 0\\
\end{align*}

In order to estimate the covariance of \(\hat\beta^*\), we use the robust
estimator of Fuller (1987):

\begin{align*}
  M_{XX} &= \frac{W' W}{n} - \hat\Sigma_u\\
  W &= \text{$n \times p$ design matrix (including intercept)}\\
  \hat\Sigma_u &= \text{covariance of measurement errors}\\
  r_i &= y_i - W_i \hat\beta\\
  \Delta_i &= W_i' r_i + \Sigma_ui \hat\beta\\
  H &= \frac{1}{n(n - p)} \sum_i \Delta_i \Delta_i'\\
  \Sigma_\beta &= M_{XX}^{-1} H M_{XX}^{-1}\\
\end{align*}

\subsection{Estimating prediction errors}

Given pre-computed weights for linear models of cis-heritable gene
expression trained in \(m\) different cohorts, we estimate \(\hat\sigma_{ui}^2\) as:

\begin{align*}
  \bar y_i &= \frac{1}{m} \sum_j y_i^{(j)}\\
  \hat\sigma_{ui}^2 &= \frac{1}{m - 1} \sum_j (y_i^{(j)} - \bar y_i)^2\\
\end{align*}

We have m=4 replicates of predicted expression (METSIM, YFS, NTR, GTEx).

Given one training cohort, we estimate \(\hat\sigma_{ui}^2\) using standard
errors from the linear model (\textbf{this is hard}).

\subsection{Simulation}

100 causal genes

\(p\) causal eQTLs per gene

Draw unlinked genotypes

PVE = 0.17 per gene

Generate phenotype as linear combination of total expression of causal genes

PVE = 0.2

Simulate four training cohorts with n = 1000. Train linear regression (since we
only have causal variants).

Estimate power by fixing n, p and simulating 10 test cohorts. Compute GWAS
p-value, eGWAS p-value, naive TWAS p-value, corrected TWAS p-value

Estimate Type I error by simulating two null phenotypes:

\begin{enumerate}
\item No link between eQTLs and phenotype: Gaussian noise
\item eQTLs are causal, but expression does not mediate phenotype (independent
effect size)
\end{enumerate}

\subsection{Data processing}

\clearpage
\printbibliography
\end{document}
